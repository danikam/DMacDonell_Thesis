\startchapter{Statistical Framework}
\label{chapter:stat}

\begin{itemize}
\item Presentation of likelihood function to be maximized in fit, and definition of each component of the likelihood.
\begin{itemize}
\item Present the strategy used to continuously interpolate changes in the binned yields going into the likelihood function associated with varying each NP.
\item Explain how NPs associated with statistical and systematic uncertainties are constrained using Gaussian constraint functions in the likelihood.
\item Present the methods used in HistFitter to handle different types of uncertainty in the fit
\begin{itemize}
\item Global normalization uncertainties
\item Correlated uncertainties of shape and normalization
\item Statistical uncertainty in each bin associated with MC simulation
\end{itemize}
\end{itemize}
\item Presentation of how normalization factors for the \wjets and \ttbar backgrounds are constrained in the CRs using a background-only fit and extrapolated to the SRs. 
\item Discussion of the evaluation of transfer factor systematics for the \wjets and \ttbar backgrounds to account for the above normalization constraint and extrapolation procedure.
\item Presentation of the discovery test to be done after unblinding $\rightarrow$ check if any fits for signal strength with unblinded data produce a statistically significant inconsistency with 0.
\item Presentation of the profiled log-likelihood ratio $q_{\mu_\text{sig}}$, and description of how $q_{\mu_\text{sig}}$ is used to calculate a p-value for the exclusion hypothesis test (in case no significant excess found in the discovery test).
\item Discussion of the use of the asymptotic formula to avoid the need to throw random pseudo-experiments when evaluating the p-value, and its regime of validity ($>\mathcal{O}(5)$ events per bin).
\item Brief discussion of the CLs method for limit setting.
\item Description of how the limits are presented in the \ms vs. $m_{Z'}$ plane.
\end{itemize}
