\startfirstchapter{Conclusion}
\label{chapter:conclusion}

The study presented in this thesis is part of a worldwide programme to search for dark matter using particle physics detectors, and focusses in particular on dark matter production at the LHC. Given the potential for yet-unconceived mechanisms by which the hypothetical interactions between dark matter and the Standard Model could occur, the search programme at the LHC emphasizes a comprehensive coverage of the possible final states that could result in the detector from dark matter production in the high-energy \(pp\) collisions, with searches guided by and interpreted using simplified models for the dark matter production mechanisms. While there could be many possible dark matter production mechanisms that would predict a signature in the \(\met+WW\) final state studied in this thesis, the construction and interpretation of the search are guided by the Dark Higgs model \cite{Duerr2017}. No statistically significant deviation was found between distributions of ATLAS collision events in the signal regions and Standard Model predictions. The search places exclusion limits on the Dark Higgs model for masses of the Dark Higgs mediator in the approximate range of 150 GeV to 350 GeV. As shown in the summary plot in Figure \ref{fig:limits_comparison}, the parameter space of the Dark Higgs model excluded by this search extends the reach of existing searches for the model performed by the ATLAS and CMS collaborations \cite{monos_had_paper,cms_monos_lep,ATL-PHYS-PUB-2019-032}, which targeted different final states.

The semileptonic \(WW(qq\ell\nu\)) final state studied by this search presented a number of opportunities compared with alternative \(WW\) decay modes to develop targeted data selections and analysis strategies that enhance the sensitivity of the search in this final state. The requirement of a single energetic lepton in the final state allows for a significant reduction of SM background processes relative to the fully hadronic channel, and the \wjets process that dominates the Standard Model background in this semileptonic final state is massively reduced by the application of a lower bound on the transverse mass between the final-state lepton and the \met. In addition, the distinct decay modes of the two \(W\) bosons enable a detailed reconstruction of the hadronically decaying \(W\). This reconstruction is facilitated in the boosted merged regime by a modification to the basic TAR algorithm \cite{ATL-PHYS-PUB-2018-012} used to reconstruct hadronic activity in the final state within one or more large-radius jets, where the modification additionally disentangles the final-state lepton from the hadronic activity. Targeted selections involving the reconstructed hadronically decaying \(W\) further reduce the background of SM processes in the search. Although the final-state neutrino prevents a full reconstruction of the Dark Higgs boson, the \minms strategy allows for an approximate reconstruction, which provides valuable shape discrimination between the DH signal model and SM background processes in the signal regions.

While the searches that target the \(WW\) final state were optimized to probe the Dark Higgs model, it is important to acknowledge that appreciable constraints on the Dark Higgs model were also obtained in the \(bb\) final state \cite{ATL-PHYS-PUB-2019-032} by re-interpreting an existing dark matter search \cite{ATLAS-CONF-2018-039} that targeted the same final state, but which was optimized to probe a different model. The impressive sensitivity of the re-interpreted search in the \(bb\) final state to the Dark Higgs model highlights the value of ensuring that searches in this \(\met+WW\) final state can also be re-interpreted in the future to constrain any alternative models that may predict a signature in the same final state. This search has been preserved for future re-interpretation using the RECAST framework \cite{Cranmer2011} developed within the ATLAS collaboration. More generally, given the vast multitude of mechanisms by which dark matter could be produced at the LHC, and the tremendous amount of human effort and computing resources involved in developing searches to probe new final states, it will be important moving forward to ensure that all new searches for dark matter can be efficiently re-interpreted to constrain alternative models, thus maximizing the potential impact of each search. 

Despite longstanding evidence from observational astronomy for the abundance of dark matter in the universe, its composition remains one of the open mysteries of modern physics. Each time that a new model is tested or new parameter space is probed, a collective step is taken towards cracking the mystery of what makes up the most abundant form of matter in the universe. 