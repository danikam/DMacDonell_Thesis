\startfirstchapter{Proposed Thesis Outline}
\label{chapter:introduction}

\begin{description}
\item[\textbf{Chapter 1}] Introduction 

\textbf{Goals}
\begin{itemize}
\item Give an overview of the current status of the SM.
\item Briefly motivate the need for dark matter in the context of astronomical observations, and present evidence in support of its hypothesized composition as a BSM particle.
\item Motivate how DM detection would improve our understanding of DM compared with existing inferences from astronomical observations.
\item Motivate the merit/complementarity of collider searches with respect to other search strategies (direct, indirect).
\item Give a picture of how this search fits into the wider DM search programme at the LHC.
\end{itemize}

\textbf{Proposed Structure}
\begin{itemize}

\item Start with 2-3 introductory paragraphs which present the search and summarize the results, as well as how they fit into the wider search for dark matter. Follow these paragraphs with a more detailed intro, with the following proposed structure:

\item Brief intro to the SM
\begin{itemize}
\item Introduce the known fundamental particles and interactions of the SM.
\item Also take this opportunity to clearly introduce SM details particularly relevant to search (eg. $W$ boson and its decay mechanisms).
\end{itemize}

\item Evidence for Dark Matter from Observational Astronomy and Cosmology
\begin{itemize}
\item Selection of some particularly interesting/compelling lines, eg. galactic rotation curves, BBN, CMB anisotropies.
\item Propose to avoid going into much detail since this topic is better covered elsewhere. 
\end{itemize}

\item Dark Matter Composition Hypotheses and Detection Strategies
\begin{itemize}
\item Briefly acknowledge the presence of theories which propose non-particle DM or DM as a SM particle (eg. MOND, primordial black holes).
\item Present evidence in favour of the more widely accepted hypothesis that DM is a fundamental BSM particle.
\item Describe what would be gained from an experimental detection/measurement of DM
\begin{itemize}
\item Solidify the case for particle DM.
\item Measure the particle properties of DM which can't be inferred from astronomical observations - mass, interaction mechanisms with SM particles, potential interactions with other BSM particles.
\end{itemize}
\end{itemize}

\item Dark Matter Search Strategies
\begin{itemize}
\item Briefly introduce direct and indirect detection approaches and their relative merits
\item More detailed intro to the collider search approach and how it complements the other search modes
\begin{itemize}
\item Sharp lower bound on accessible DM masses suffered by direct detection searches due detector noise threshold is not an issue for collider searches. Neither is the neutrino floor.
\item Possibility to tailor searches to specific hypothetical DM production models by tuning selections $\rightarrow$ emphasize this point, since it's important for this search.
\item If evidence for DM found by any method (collider or otherwise), colliders can offer a superior means of pursuing dedicated measurements of its properties compared with other search modes.
\end{itemize}
\end{itemize}

\item Search for the Dark Higgs model in the context of the wider accelerator search programme
\begin{itemize}
\item Give an idea of the breadth of accelerator DM search program.
\begin{itemize}
\item Resonance searches
\item mono-X searches
\item Model-independent and model-dependent approaches $\rightarrow$ briefly present range of model completeness, from EFT through simplified to complete.
\item Cover searches at hadron and $e^+e^-$ colliders and fixed target experiments.
\end{itemize}
\item Discuss the relative merits of searching for simplified models (bridge gap between EFT and complete theories, avoid overtuning problems inherent with complete models, facilitate adequate coverage of plausible DM production processes).
\item Briefly introduce the Dark Higgs model, and clearly identify it as a simplified model (details to be fleshed out in Chapter 2) $\rightarrow$ emphasize that this search is sensitive to heavy DM ($\gtrsim$60 GeV), with the requirement that $m_\chi>\frac{1}{2}\ms$ (to prevent the $s\rightarrow\chi\chi$ decay mode).
\end{itemize}
\end{itemize}

\item[\textbf{Chapter 2}] The Dark Higgs Model

\textbf{Goals}
\begin{itemize}
\item Present the theoretical motivation for the Dark Higgs model.
\item Explain what makes the signature unique from other mono-X searches.
\item Motivate why we look for the model in the $s\rightarrow WW$ channel.
\end{itemize}

\textbf{Proposed Structure}
\begin{itemize}
\item Theoretical description of the model
\begin{itemize}
\item Introduce it in the context of the wider class of dark sector models.
\item Describe the production mechanism, including leading Feynman diagrams.
\item Specify couplings to both SM and dark sector particles. 
\item Emphasize that $m_\chi>\frac{1}{2}\ms$ required to obtain signature in detector (otherwise $s\rightarrow\chi\chi$ decay would dominate).
\end{itemize}
\item Theoretical motivation for the model
\begin{itemize}
\item Dark Higgs needed to generate masses of dark sector particles.
\item Need for creation and annihilation mechanism between SM and DM in early universe (thermal freeze-out hypothesis) motivates mixing between SM Higgs and dark sector Higgs.
\end{itemize}
\item Discussion of the model's signature in the ATLAS detector (boosted SM pair recoiling against \met), and why the boosted topology is unique compared with generic mono-X searches in which the `X' is produced via ISR.
\item Discussion of available search channels - mono-s(bb), mono-s(WW), mono-s(ZZ), mono-s(hh).
\begin{itemize}
\item Overview of existing searches for the Dark Higgs model - re-interpreted mono-h(bb), hadronic mono-s(WW). Could also mention ongoing dedicated mono-s(bb) search.
\item Identify the \ms regime in which the $s\rightarrow WW$ decay mode dominates in sensitivity.
\end{itemize}
\end{itemize}

\item[\textbf{Chapter 3}] Introduction to the LHC and the ATLAS detector

\textbf{Goals}
\begin{itemize}
\item Provide a general intro to the LHC, focussing on its world-leading energy and luminosity, and their implications for enabling novel physics searches.
\item Introduce the ATLAS detector, and describe its sub-detectors in sufficient detail to give context for the later presentation of object definitions and event selections. 
\end{itemize}

\textbf{Proposed Structure}
\begin{itemize}
\item Brief introduction to the LHC
\begin{itemize}
\item Location, circumference, CoM energy, luminosity. Try to give some intuitive meaning to the CoM energy and luminosity. Emphasize that the large CoM collision energy makes it possible to probe models of new physics with heavy mediators, and the high luminosity is needed to produce sufficient statistics to search for statistically significant discrepancies between data and MC which could be indicative of new physics. 
\item Four interaction points, each surrounded by a detector $\rightarrow$ give a brief introduction (~1 sentence) of the other three detectors to give an idea of how ATLAS fits into the wider LHC physics programme.
\item Brief discussion of the parton model and PDFs, making the point that it's the partons colliding at the LHC, and that the colliding partons and their fraction of momentum carried are probabilistic and described by the proton PDF for a given momentum transfer scale. This will be useful both for the discussion of \met, and in case I want to mention our study of the potential benefit of binning in lepton charge.
\end{itemize}
\item Introduction to the ATLAS detector, giving an idea of its scale and significance as one of the two general purpose particle detectors at the LHC (enables a wide range of physics measurement and search programmes; complementarity with CMS).
\item Inner detector $\rightarrow$ discussion of charged particle tracking will be relevant for later description of TAR jet reconstruction (may want to point that out already).
\item Calorimeters $\rightarrow$ emphasize distinction between small- and large-radius jets, and between electromagnetic and hadronic showers. 
\begin{itemize}
\item Talk about electron detection after/during the description of the EM calorimeter (should have all needed info since the inner tracker has already been discussed). 
\end{itemize}
\item Muon spectrometer for muon detection $\rightarrow$ emphasize that muons pass through the other sub-detectors. 
\item \met 
\begin{itemize}
\item Define \met here, now that all sub-detectors have been described. This intro to \met will be needed for the discussion of the \met trigger in the next section.
\item Shouldn't need to go into too much detail on the objects involved in \met reconstruction, since this will be covered in more detail in Chapter 5.
\end{itemize}
\item Trigger system 
\begin{itemize}
\item Discuss the relatively enormous cross sections of soft QCD processes $\rightarrow$ emphasize that much of the trigger design and data selections are devoted to reducing this soft QCD background to focus on the rarer physics processes of interest for measurements/searches. 
\item Otherwise, focus on the \met trigger, and mention that it only uses info from the calorimeter (relevant for later presentation of our use of the \met OR single muon trigger).
\end{itemize}
\end{itemize}

\item[\textbf{Chapter 4}] Modelling the Dark Higgs Model and SM Background Processes

\textbf{Goals}
\begin{itemize}
\item Introduce the concept of Monte Carlo and its application to modelling collision processes at the LHC. 
\item Explain why MC simulation of SM background and signal processes is needed to perform searches with ATLAS data.
\item Briefly present the use of Geant for modeling the passage of particles through the ATLAS detector. Try to give an idea of its complexity, without going into too much detail.
\item Present the MC simulation of the Dark Higgs signal model and the relevant SM background processes in enough detail to help the reader digest subsequent plots of MC distributions, data-MC comparisons and discussions of fluctuations and uncertainties due to limited MC stats.  
\end{itemize}

\textbf{Proposed Structure}
\begin{itemize}
\item Generic introduction to the Monte Carlo method as a modeling technique.
\item Discuss of the application of Monte Carlo for modeling particle collision processes at the LHC. 
\item Present the concept of cross section and luminosity, and discuss the need to scale MC simulated data to the integrated LHC data luminosity. 
\item Broadly describe how the collected data can be compared with MC simulated signal and SM background processes to search for new physics.
\item Presentation of signal grid and simulation details (eg. leading Feynman diagrams, use of FullSim, inclusion of additional final-state jet, etc.).
\item Description of the background processes considered and why they make it into our analysis regions.
\begin{itemize}
\item Very brief description of how each background process is modelled (presumably don't need to get quite as in-depth as the description in section 6 of the support note). 
\end{itemize}
\end{itemize}

\item[\textbf{Chapter 5}] Object Definitions, Trigger and Event Selection

\textbf{Goals}
\begin{itemize}
\item Introduce the detector objects needed to define our event selection.
\item Present the trigger combination used, and motivate the need for inclusion of the single muon trigger.
\item Provide a motivation for and detailed description of the selections used to define the signal regions for the search.
\item Discuss the value of using CRs to constrain the normalizations of leading background processes.
\item Present and motivate the cut modifications (relative to SR selections) used to define the CRs.
\end{itemize}

\textbf{Proposed Structure}
\begin{itemize}
\item Object Definitions (propose to roughly follow structure presented in support note)
\begin{itemize}
\item Electrons
\item Muons
\item Small-radius \aktfour jets
\item TAR jets
\item Overlap Removal
\item \met
\item Dark Higgs candidate mass \ms
\end{itemize}

\item Trigger
\begin{itemize}
\item State the trigger combination used for the search (\met OR single muon)
\item Show trigger efficiency curves for \met only and \met OR single muon to show that the (OR single muon) is a necessary addition in the muon channel to achieve 100\% sensitivity.
\item Explain why \met trigger alone is insufficient in the muon channel due the exclusive use of calorimeter information by the ATLAS \met trigger.
\end{itemize}

\item Event Selections
\begin{itemize}
\item High level discussion of why we apply event selections, and goals for optimal signal region definition.
\begin{itemize}
\item Broadly: maximize predicted signal content and minimize simulated background content, while maintaining sufficient MC and data statistics to enable a meaningful comparison between MC and data.
\item Prioritize optimization of signal points near the edge of expected search sensitivity. 
\item Keep signal region blind during optimization to avoid biasing selection.
\end{itemize}
\item Introduce variables used for event selection. Distinguish between variables that are optimized (eg. \mtlepmet) vs. fixed (eg. 1-lepton requirement) during optimization.
\item Present concept and implementation of signal region optimization strategy.
\item High-level discussion of why we define CRs to constrain normalizations of dominant \wjets and \ttbar backgrounds.
\begin{itemize}
\item Provides data-driven normalization constraint which can be extrapolated to the signal region (more details on extrapolation procedure in Chapter 7)
\item Reduces the impact of (and reliance on) theoretical uncertainties involved in simulating the correct normalizations for these backgrounds. Emphasize the difficulty involved with assigning reliable theoretical uncertainties, and hence the value of using data-driven constraints.
\end{itemize}
\item Summary of design goals for control region
\begin{itemize}
\item High purity of background of interest.
\item Orthogonal to SR.
\item Phase space kinematically similar to SR.
\item Signal contamination negligible compared with uncertainty of total background yield.
\end{itemize}
\item Present the \wjets control region, and motivate the \dR reversal used to define it.
\item Present the \ttbar control region, and motivate the \bjet veto reversal used to define it.
\item Present the additional modifications that were needed in the \merged category to optimize the CR definitions
\begin{itemize}
\item Reducing the lower bound on \metsig to boost stats.
\item Increasing the lower bound on \dR in the \wjets CR to reduce the signal contamination to an acceptable level.
\end{itemize}
\item Summary of all analysis regions.
\end{itemize}
\end{itemize}

\item[\textbf{Chapter 6}] Systematic Uncertainties

\textbf{Goals}
\begin{itemize}
\item Explain why it's important to identify all sources of systematic uncertainty and evaluate their effects on final analysis yields. 
\item Clarify the distinction between theoretical and experimental sources of systematic uncertainty, and the differences in how they're evaluated.
\item Present all sources of experimental and theoretical uncertainty considered, how each is evaluated, and their impacts on final yields.
\end{itemize}

\textbf{Proposed Structure}
\begin{itemize}
\item General discussion of how systematic uncertainties can bias the simulation of physical objects, ultimately resulting in a systematic mis-modelling of the yields and shapes of simulated processes in our analysis regions.
\item Discussion of how systematic uncertainties are evaluated using the up/down shift in the yield within each bin obtained by varying the given source of systematic uncertainty.  
\item Clearly distinguish between theoretical and experimental sources of systematic uncertainty. Discuss the relative difficulty often involved with assigning reliable theoretical uncertainties (can be challenging to assess realistic bounds on the possible range of values for theoretical inputs, whereas experimental bounds are generally more measurable). 
\item Discussion of each major source of experimental systematics, how they're evaluated and some plots and tables showing shifts in yield arising from the dominant sources of experimental systematics.
\item Discussion of the sources of theory systematics considered (QCD scale, PDF, $\alpha_s$, PS) and how each is evaluated for signal and background.
\item Explanation of the special treatment of the triboson theory uncertainty (i.e. setting it to 100\%) due to the absence of any appropriate alternate generator sample.
\item Plots and tables showing the yield shifts and relative sizes of each source of theory uncertainty for signal and major backgrounds.
\item Comparison of the relative impact of statistical vs. systematic uncertainties in each analysis region, to identify where we're stats- vs. systematics-dominated.
\end{itemize}

\item[\textbf{Chapter 7}] Statistical Framework

\textbf{Goals}
\begin{itemize}
\item Explain how the likelihood to be maximized in the fit is constructed.
\item Explain how the evaluated uncertainties and the \wjets and \ttbar normalization factors are incorporated into the likelihood as nuisance parameters.
\item Describe the method used to extrapolate \wjets and \ttbar normalization factors from the control region to the signal region.
\item Present the binning strategy used in the analysis regions, and briefly describe how the bin widths were optimized in the \merged SR.
\end{itemize}

\textbf{Proposed Structure}
\begin{itemize}
\item Presentation of likelihood function to be maximized in fit, and definition of each component of the likelihood.
\begin{itemize}
\item Present the strategy used to continuously interpolate changes in the binned yields going into the likelihood function associated with varying each NP.
\item Explain how NPs associated with statistical and systematic uncertainties are constrained using Gaussian constraint functions in the likelihood.
\item Present the methods used in HistFitter to handle different types of uncertainty in the fit
\begin{itemize}
\item Global normalization uncertainties
\item Correlated uncertainties of shape and normalization
\item Statistical uncertainty in each bin associated with MC simulation
\end{itemize}
\end{itemize}
\item Presentation of how normalization factors for the \wjets and \ttbar backgrounds are constrained in the CRs using a background-only fit and extrapolated to the SRs. 
\item Discussion of the evaluation of transfer factor systematics for the \wjets and \ttbar backgrounds to account for the above normalization constraint and extrapolation procedure.
\item Presentation of the discovery test to be done after unblinding $\rightarrow$ check if any fits for signal strength with unblinded data produce a statistically significant inconsistency with 0.
\item Presentation of the profiled log-likelihood ratio $q_{\mu_\text{sig}}$, and description of how $q_{\mu_\text{sig}}$ is used to calculate a p-value for the exclusion hypothesis test (in case no significant excess found in the discovery test).
\item Discussion of the use of the asymptotic formula to avoid the need to throw random pseudo-experiments when evaluating the p-value, and its regime of validity ($>\mathcal{O}(5)$ events per bin).
\item Brief discussion of the CLs method for limit setting.
\item Description of how the limits are presented in the \ms vs. $m_{Z'}$ plane.
\end{itemize}

\item[\textbf{Chapter 8}] Results

\textbf{Goals}
\begin{itemize}
\item Show validation plots/tables for both background-only and sample signal+bkg fit
\item Show signal+bkg fit results
\item Claim discovery or set limits, as appropriate.
\end{itemize}

\textbf{Proposed Structure}
\begin{itemize}
\item Present the background-only fit results (before/after yield plots and tables, pull plots and correlation plot for NPs) for validation.
\item Present validation plots (same as above) for a signal+bkg fit at a sample signal point.
\item Plot fitted signal strength and significance over the full signal grid (in the \ms vs. $m_{Z'}$ plane)
\item If no significant excess observed, plot observed limit overlaid on projected sensitivity in \ms vs. $m_{Z'}$ plane.
\end{itemize}

\item[\textbf{Chapter 9}] Conclusion

Dependent on results.

%\item[\textbf{Appendix A}] Presentation of theoretical concepts that don't really fit in nicely anywhere in the main thesis
%\begin{itemize}
%\item Theoretical description of cross section and luminosity
%\end{itemize}

\end{description}
