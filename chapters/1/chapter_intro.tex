\startfirstchapter{Introduction}
\label{chapter:introduction}

Start with 2-3 introductory paragraphs which present the search and summarize the results, as well as how they fit into the wider search for dark matter. 

\section{Introduction to the Standard Model}

\begin{itemize}
\item Introduce the known fundamental particles and interactions of the SM.
\item Also take this opportunity to clearly introduce SM details particularly relevant to search (eg. $W$ boson and its decay mechanisms).
\item Include a discussion of real vs. virtual particles, and on- vs. off-shell production.
\end{itemize}

\section{Evidence for Dark Matter from Observational Astronomy and Cosmology}

\begin{itemize}
\item Selection of some particularly interesting/compelling lines, eg. galactic rotation curves, BBN, CMB anisotropies.
\item Propose to avoid going into much detail since this topic is better covered elsewhere. 
\end{itemize}

\section{Dark Matter Composition Hypotheses}

\begin{itemize}
\item Briefly acknowledge the presence of theories which propose non-particle DM or DM as a SM particle (eg. MOND, primordial black holes).
\item Present evidence in favour of the more widely accepted hypothesis that DM is a fundamental BSM particle.
\item Describe what would be gained from an experimental detection/measurement of DM
\begin{itemize}
\item Solidify the case for particle DM.
\item Measure the particle properties of DM which can't be inferred from astronomical observations - mass, interaction mechanisms with SM particles, potential interactions with other BSM particles.
\end{itemize}
\end{itemize}

\section{Dark Matter Search Strategies}
\begin{itemize}
\item Briefly introduce direct and indirect detection approaches and their relative merits
\item More detailed intro to the collider search approach and how it complements the other search modes
\begin{itemize}
\item Sharp lower bound on accessible DM masses suffered by direct detection searches due detector noise threshold is not an issue for collider searches. Neither is the neutrino floor.
\item Possibility to tailor searches to specific hypothetical DM production models by tuning selections $\rightarrow$ emphasize this point, since it's important for this search.
\item If evidence for DM found by any method (collider or otherwise), colliders can offer a superior means of pursuing dedicated measurements of its properties compared with other search modes.
\end{itemize}
\end{itemize}

\subsection{Searching for Dark Matter at Particle Accelerators}

\begin{itemize}
\item Give an idea of the breadth of accelerator DM search program.
\begin{itemize}
\item Resonance searches
\item mono-X searches
\item Model-independent and model-dependent approaches $\rightarrow$ briefly present range of model completeness, from EFT through simplified to complete.
\item Cover searches at hadron and $e^+e^-$ colliders and fixed target experiments.
\end{itemize}
\item Discuss the relative merits of searching for simplified models (bridge gap between EFT and complete theories, avoid overtuning problems inherent with complete models, facilitate adequate coverage of plausible DM production processes).
\item Briefly introduce the Dark Higgs model, and clearly identify it as a simplified model (details to be fleshed out in Chapter 2) $\rightarrow$ emphasize that this search is sensitive to heavy DM ($\gtrsim$60 GeV), with the requirement that $m_\chi>\frac{1}{2}\ms$ (to prevent the $s\rightarrow\chi\chi$ decay mode).
\end{itemize}

