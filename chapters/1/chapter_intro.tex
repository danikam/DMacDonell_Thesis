\startfirstchapter{Introduction}
\label{chapter:introduction}

Start with 2-3 introductory paragraphs which present the search and summarize the results, as well as how they fit into the wider search for dark matter. 

\section{Introduction to the Standard Model}

The Standard Model (SM) describes all known elementary particles and three of the four known forces by which they interact with one another - the strong, the electromagnetic (EM), and weak forces. The theory of general relativity, which describes the gravitational force, has yet to be incorporated into the SM. 

The known particles, illustrated in Figure \ref{fig:standard_model} are divided into two classes known as ``fermions" and ``bosons" on the basis of an intrinsic form of angular momentum known as ``spin". Fermions carry spin \(\frac{1}{2}\) and bosons carry integer spin. 

The specific forces by which particles in the SM interact with one another are determined by the charge(s) that they carry. Particles carrying electric charge interact with other particles carrying this charge via the EM force. Similarly, particles carrying weak and colour charge interact via the weak and strong forces, respectively. 

Each fermion has a corresponding anti-particle with the same mass, but with opposite values of the charges carried - for example, the electron carries negative electric charge and its antiparticle, the positron, carries positive electric charge. 

\begin{figure}[H]
	\centering
	\includegraphics[width=0.7\textwidth]{Figures/1/StandardModel.pdf}
	\caption[]{Names and fundamental properties of particles in the Standard Model}
	\label{fig:standard_model}
\end{figure}

\subsection{Fermions}

Fermions are further sub-divided into leptons and quarks, depending on the charges they carry, and hence the forces by which they interact. There are three known generations of fermions, labelled I, II and III in Figure \ref{fig:standard_model}, each with significantly higher mass than the last. Each generation contains a pair of quarks and a pair of leptons, along with their associated antiparticles. The quark pair consists of one ``up-type" quark with positive electric charge and one ``down-type" with negative charge. The lepton pair consists of one charged lepton and one charge-neutral ``neutrino". 

Leptons carry electric charge and weak isospin, and as a result interact with one another and with other particles carrying these charges via the EM and weak forces, respectively.  

Quarks also carry electric charge and weak isospin, and additionally carry colour charge. The colour charge allows quarks to interact via the strong force, such that quarks interact by all three forces described by the SM. Unlike charged leptons, which carry an electric charge of \(\pm1\), quarks carry fractional electric charged; up-type quarks carry a charge of \(+\frac{2}{3}\) and down-type carry a charge of \(-\frac{1}{3}\).

Due to an effect known as ``colour confinement", quarks cannot exist as stable particles in isolation, and must instead combine with other quarks to form stable ``colour-neutral" states called ``hadrons". The two major forms of hadrons are ``mesons" formed by a quark-antiquark pair and ``baryons" formed by three quarks. Due to the strength of the strong interaction, there is a relatively high probability that particle production and decay initiated by high energy $pp$ collisions at the LHC will proceed via strong force interactions compared with other forces. As a result, the vast majority of decay products observed in the ATLAS detector are cascades of hadronic interactions in the calorimeter referred to as ``jets" \footnote{See Section \ref{sec:had_calo} for a more detailed discussion of jets in the hadronic calorimeter.} which are initiated by hadrons produced in the collisons.

\subsection{Bosons}

Bosons in the SM are divided into ``gauge bosons" and ``scalar bosons". The gauge bosons are spin 1 force carriers which mediate interactions between particles. The photon mediates EM interactions between electrically charged particles. The gluon mediates the strong interaction between quarks. Unlike photons which are charge-neutral, the gluon itself carries colour charge, which allows it to self-interact via the strong force. The weak force is mediated by three particles: the electrically neutral Z boson, and two W bosons (W$^\pm$) with opposite electric charges of $\pm$1. 

Scalar bosons are defined as spin 0 particles. There is only one known scalar boson in the SM, namely the Higgs boson (or, simply, the ``Higgs"). Particles in the SM acquire mass via their interaction with the Higgs field. As such, the Higgs only interacts with massive SM particles, which includes all particles except the photon and the gluon. The more massive the particle, the greater its interaction strength - i.e. probability of interaction - with the Higgs. Neutrinos are a possible exception; there is at present no mechanism in the SM by which neutrinos could interact with the Higgs field, so the origin of their tiny masses remains an open question.  

\subsection{Mathematical Formulation of the Standard Model}

The SM is formulated mathematically as a quantum field theory, in which particles of the SM are represented as excitations of quantum fields of spacetime \(x\). The mathematical formulation of the SM is presented in detail in standard texts \cite{griffiths_2008, SM_intro}, and briefly summarized in this section, with focus placed on aspects that are relevant to later discussions in this thesis.

\subsubsection{Lagrangian Densities}

As in classical field theories, the quantum fields of the SM and their interactions are powerfully described by the formalism of Lagrangian densities, which are functions of the quantum fields and their derivatives. For example, interactions between photons and electrically charged fermions are described in quantum electrodynamics (QED) by the following Lagrangian density term:

\begin{equation}
\label{eq:qed_interaction}
\mathcal{L}_\text{QED, interaction} = -q\psi^\dagger(x)\gamma^0\gamma^\mu\psi(x) A_\mu(x)
\end{equation}

\noindent where \(\psi(x)\) represents the spinor field of the spin-\(\frac{1}{2}\) fermions in the SM, \(A_\mu(x)\) represents the vector field of the massless spin-1 photon and \(\gamma^\mu\) are the Dirac matrices \cite{griffiths_2008}. The index \(\mu\) runs over the four spacetime coordinates. The factor \(q\) represents the charge of the fermion involved in the interaction, and its value - \(\pm1\) for charged leptons, \(+\frac{2}{3}\) (\(-\frac{1}{3}\)) for up (down) type quarks and 0 for neutrinos - determines the strength of the interaction. 

\subsubsection{Symmetries and Groups Theory Description}

Symmetries in the Lagrangian densities are described in the language of group theory by classifying the fundamental interactions into gauge groups which describe their symmetries. For example, QED exhibits a symmetry under local phase transformation, described by the unitary local gauge group U(1). This means that the Lagrangian \(\mathcal{L}_\text{QED}\) is invariant under the multiplication of the fermion spinor \(\psi\) by a unitary \(1\times1\) matrix U (\(U^\dagger U=1\)), which represents a local phase transformation \(U = e^{i\theta(x)}\) where \(\theta(x)\) can be any function of the spacetime coordinates \(x\). This symmetry is ensured by the inclusion of the vector field \(A_\mu(x)\) in the QED Lagrangian, which is identified with the physical photon. Because it ensures invariance under the U(1) gauge group, the vector field \(A_\mu(x)\) is referred to as a ``gauge field", and the corresponding boson (the photon) as a ``gauge boson". The symmetries in the SM are described by the direct product\footnote{General definitions of group theory can be found, for example, in Section 1.1 of Ref. \cite{costa2012symmetries}.} of U(1)\(\times\)SU(2)\(\times\)SU(3) gauge groups. 

\subsubsection{Quantum Chromodynamics}

The theory of quantum chromodynamics (QCD) \cite{qcd_2007}, which describes the strong interactions mediated by gluons between particles with colour charge (quarks and gluons) is described by the SU(3) gauge group. The quarks are represented by a three-component vector of spinors: \(\psi_c = \{\psi_r, \psi_b, \psi_g\}\), where the subscripts refer to the three colours that constitute the charges of the strong interaction. The QCD Lagrangian (see eg. Eq. 10.88 in Ref. \cite{griffiths_2008}) is symmetric under a transformation of the quark spinor \(\psi_c\) by a \(3\times3\) SU(3) matrix which is unitary with determinant 1. The SU(3) symmetry is ensured by the presence of an eight-component set \(\boldsymbol{\boldsymbol{A}^\mu}\) of vector gauge fields in the QCD Lagrangian associated with massless gauge bosons. The massless gauge bosons are identified as the eight physical gluons, where each of the eight physical gluons possesses a unique superposition of \({rgb}\) colour states \cite{griffiths_2008}.

\subsubsection{Electoweak Theory and the Higgs Mechanism}

The mathematical descriptions of the weak and EM forces are unified into a single ``electroweak" \cite{electroweak_2012} theory whose symmetries are described by the SU(2)\(\times\)U(1) product of gauge groups. 

Yang-Mills theory \cite{yang_mills_1954} shows that a set of three vector gauge fields associated with massless gauge bosons are needed to satisfy the SU(2) symmetry, and a fourth massless vector gauge boson is needed to satisfy the U(1) symmetry. The three-component set of vector bosons required to satisfy the SU(2) symmetry are identified as the ``weak isospin triplet" \(\boldsymbol{W}\), and the single gauge boson needed to satisfy the U(1) symmetry as the singlet \(B\). 

The SU(2)\(\times\)U(1) symmetry of the SM Lagrangian does not admit mass terms of the form \(m_X^2X^\dagger X\), where X is an arbitrary field. Masses of the physical \(W^\pm\) and \(Z\) bosons, and all other massive particles in the SM are generated by the ``Higgs mechanism" \cite{HiggsTheory1,HiggsTheory2,HiggsTheory3}, which adds the following term to the SM Lagrangian:

\begin{equation}
\label{eq:higgs_lagrangian}
\mathcal{L}_\text{Higgs} = (D^\mu H)^\dagger(D_\mu H) - V(H)
\end{equation}

\noindent where \(H\) is a complex scalar field whose symmetries are described by the SU(2) group:

\begin{equation}
H = \frac{1}{\sqrt{2}}
\begin{pmatrix}
0 \\
h+v
\end{pmatrix}
\end{equation}

\noindent where \(h(x)\) is interpreted as the scalar field of the physical Higgs boson, and \(v\) is the vacuum expectation value. With \(H\) in this form, \(\mathcal{L}_\text{Higgs}\) is described by the U(1) symmetry group but not the SU(2) group, and is thus said to ``break" the electroweak symmetry SU(2)\(times\)U(1) to the QED gauge symmetry U(1).

The covariant derivative \(D_\mu H\) in Eq. \ref{eq:higgs_lagrangian} takes the form:

\begin{equation}
\label{eq:D_muH}
D_\mu H = \big(\partial_\mu + i\frac{1}{2}g\sigma_k W^k_\mu+i\frac{1}{2}g'B_\mu\big)H
\end{equation}

\noindent where \(\sigma_k\) are the Pauli matrices, and \(g\) and \(g'\) are the coupling constants between the Higgs field and the \(\boldsymbol{W}\) and \(\boldsymbol{B}\) fields, respectively.

The Higgs potential \(V(H)\) takes the form 

\begin{equation}
V(H) = -\mu^2H\dagger H + \lambda(H^\dagger H)^2
\end{equation}

\noindent where the second term describes quartic self-interactions of the Higgs field.

The emergence of the massive physical \(W^\pm\), \(Z\) bosons and the massless photon comes from the interaction between the 

 can be seen by expanding Eq. \ref{eq:D_muH}, considering only the terms involving the vacuum expectation value \(v\):

\begin{equation}
\label{eq:higgs_expanded}
\mathcal{L}_\text{Higgs} = \frac{v^2}{8}\Big[g^2\big((W^1_\mu)^2+(W_mu^2)^2\big) + (gW^3_\mu-g'B_\mu)^2\Big] + ...
\end{equation}

With the physical vector boson fields and masses defined as:

\begin{equation}
\begin{split}
W^\pm_\mu & \equiv \frac{1}{2}(W_\mu^1 \mp W_\mu^2) \phantom{xxxxxxxxxlxx}\text{ with mass }\phantom{xxx} m_W=\frac{gv}{2} \\
Z_\mu & \equiv \frac{1}{\sqrt{g^2+g'^2}}(gW_\mu^3-g'B_\mu) \phantom{xxx}\text{ with mass }\phantom{xxx} m_Z = \frac{v}{2}\sqrt{g^2+g'^2} \\
A_\mu & \equiv \frac{1}{\sqrt{g^2+g'^2}}(g'W^3_\mu+gB_\mu) \phantom{xxx}\text{ with mass }\phantom{xxx} m_A = 0
\end{split}
\end{equation}

It can be readily confirmed that Eq. \ref{eq:higgs_expanded} takes the form \(\mathcal{L}_\text{Higgs} = \big[(W^\pm)_\mu^\dagger(W^\pm)^\mu + Z_\mu^\dagger Z^\mu + A_\mu^\dagger A^\mu\big] + ...\). Masses of fermions are likewise generated by so-called Yukawa couplings \cite{weinberg_1967} between the fermion and Higgs fields. The dark Higgs model used to optimize and interpret the DM search presented in this thesis postulates that particles in the dark sector would acquire their masses by means of their interaction with the dark Higgs field \(S\), as discussed in Chapter \ref{chapter:dh_model}. 

\subsection{Particle Decay and Lifetime}

The lowest-mass ``first-generation" quarks and leptons that comprise column I in Figure \ref{fig:standard_model}, along with the massless photons and gluons, are the only stable particles in the SM. All other particles are unstable, and will decay to less-massive particles after they are produced. The decay of an unstable particle occurs randomly with respect to the time elapsed since the particle was produced. However, this random process is governed by Poisson statistics, and the likelihood that an unstable particle will remain after some period \(t\) decays exponentially, with a mean lifetime \(\tau\) in the particle's rest frame:

\begin{equation}
\label{eq:particle_decay}
P(t) = e^{\frac{t}{\tau}}
\end{equation}

\noindent where the decay rate \(\Gamma\) is the inverse of the mean lifetime.  

\subsection{Collision and Decay Processes at Colliders}
\label{sec:col_decay_procs}

The high-energy counter-rotating proton beams at the LHC are brought into head-on collisions at four interaction points around the ring, each of which is surrounded by a detector\footnote{See Chapter \ref{chapter:lhc_atlas} for a detailed discussion of the LHC and the detectors which surround the four interaction points.}. At each interaction point, constituents of the colliding protons known as ``partons"\footnote{See Section \ref{sec:parton_model} for an introduction to the parton model.} can pair annihilate to form observable collision products via one or more ``virtual mediators"\footnote{See Section \ref{sec:virtual_particles} for a discussion of virtual particles.}, and the collision products are subsequently measured by the detector. 

Each process that describes a mechanism by which partons may annihilate to form observable products has a certain probability of taking place relative to other possible annihilation and production processes. The probability that a given process will take place is quantified by its ``cross section" \(\sigma\). The beam luminosity \(\mathcal{L}\) relates the rate of collisions \(\frac{dN}{dt}\) which proceed via a given process to the cross section of the process:

\begin{equation}
\frac{dN}{dt} = \mathcal{L}\sigma
\end{equation}

The luminosity can be integrated over a period of time \(t_1\) to \(t_2\), such that the total number of events expected to be produced via a process with cross section \(\sigma\) over the given period is related to the ``integrated luminosity" \(\mathcal{L}_\text{int}\) by:

\begin{equation}
\label{eq:integrated_lumi}
N = \sigma\int_{t_1}^{t_2}\mathcal{L}(t)dt = \sigma\mathcal{L}_\text{int}
\end{equation}

\subsubsection{Feynman Diagrams}

The interaction mechanisms by which observable collision products are produced from the annihilation of two partons can be represented by Feynman diagrams, which are described in detail in Chapter 2 of Ref. \cite{griffiths_2008} and summarized here. As an example, the Feynman diagram for the Drell Yan process in which a \(q\bar{q}\) pair annihilate to form a lepton pair \(\ell\bar{ell}\) via a virtual photon \(\gamma^{*}\) or Z boson \(Z^{*}\) mediator is shown in Figure \ref{fig:drell_yan}. 

The particles involved in the interactions are represented as lines in a Feynman diagram, with different particle types represented by different line styles - fermions are generally represented by solid straight lines, and bosons (with the exception of gluons) are generally represented by wavy lines. Particle interactions are represented by vertices at which the lines in the diagram intersect. The \(q\bar{q}\) annihilation to form the virtual \(\gamma^{*}/Z^{*}\) mediator is represented in Figure \ref{fig:drell_yan} by the vertex at which the \(q\) and \(\bar{q}\) fermion lines meet the \(\gamma^{*}/Z^{*}\) boson line, and the subsequent decay of the  \(\gamma^{*}/Z^{*}\) to \(\ell\bar{\ell}\) is represented by the vertex to the right at which the \(\gamma^{*}/Z^{*}\) line meets the \(\ell\) and \(\bar{\ell}\) lines. Note that time flows horizontally from left to right in Feynman diagrams, so the colliding \(q\bar{q}\) pair are shown on the left and the observable decay products \(\ell\bar{\ell}\) on the right.

\begin{figure}[hp]
	\centering
%	\includegraphics[width=0.95\textwidth]{Figures/2/Fey1.pdf}
		\begin{tikzpicture}
			\begin{feynman}

		 		\vertex (a1);
		 		\vertex at ($(a1) + (0cm, -3cm)$) (b1);
		 		\vertex at ($(a1) + (1cm, -1.5cm)$) (c1); %gamma/Z
		 		\vertex at ($(c1) + (2cm, 0cm)$) (c2); %gamma/Z
				\vertex at ($(c2) + (1cm, -1.5cm)$) (d1);
				\vertex at ($(c2) + (1cm, 1.5cm)$) (d2);

		 		\diagram* {
		 		  {[edges=fermion]
		 		    (b1) -- [edge label=\(q\)]( c1) -- [edge label=\(\bar{q}\)](a1),
				    (d1) -- [edge label=\(\bar{\ell}\)]( c2) -- [edge label=\(\bar{\ell}\)](d2),
		 		  },
		 		  (c1) -- [boson, edge label=\(\gamma^{*}/Z^{*}\)] (c2),
		 		};
		 	\end{feynman}
		 \end{tikzpicture}
	\caption{Feynman diagram for the Drell Yan process.}
	\label{fig:drell_yan}
\end{figure}

\subsubsection{Virtual Particles}
\label{sec:virtual_particles}

In general, ingoing and outgoing lines in a Feynman diagram represent real observable particles, and internal lines represent so-called virtual particles. Virtual particles are not observable, but are rather a representation of the mechanism involved with producing the observable final state products. Importantly, virtual particles are not in general produced with the mass of their corresponding real particle, but can in principle take on whatever mass is needed to satisfy energy and momentum conservation at each interaction vertex that they are involved with. However, the more ``off-shell" the mass of the virtual particle, meaning the more it differs from the mass of the corresponding real particle, the lower is the production cross section \(\sigma(m^{*})\) with which the process would be expected to proceed for the given mass \(m^{*}\) of the virtual particle required to satisfy energy-momentum conservation. This relationship is described quantitatively by the Breit-Wigner formula \cite{breit_wigner}:

\begin{equation}
\label{eq:breit_wigner}
\sigma(m^{*}) \propto \frac{1}{(m^{*}-m_0)^2 + \frac{\Gamma_E^2}{4}}
\end{equation}

\noindent where \(m_0\) is the ``on-shell" mass of the corresponding real particle, and \(\Gamma\) is the total decay width of the real particle. The Breit-Wigner formula describes a peak centred at \(m_0\) with width \(\Gamma_E\). The lifetime of the corresponding particle is related to the width of its Breit-Wigner resonance by \(\tau = \frac{\hbar}{\Gamma_E}\).

\subsubsection{Matrix Element}

%In order to study collision events measured by particle detectors at the LHC, it is important to calculate the rate at which the detector would be expected to measure collision events which proceed by different processes, such as the Drell-Yan process shown in Figure \ref{fig:drell_yan}. 

The cross section associated with a process of particle production from \(pp\) collisions at the LHC, such as the Drell-Yan process shown in Figure \ref{fig:drell_yan}, is in general proportional to an integral of the squared matrix element \(|\mathcal{M(\boldsymbol{x}, \boldsymbol{\alpha})}|^2\):

\begin{equation}
\label{eq:matrix_element}
\sigma \propto \int|\mathcal{M(\boldsymbol{x}, \boldsymbol{\alpha})}|^2 d\boldsymbol{x} 
\end{equation}

\noindent where the quantities \(\boldsymbol{x}\) describe the dynamics (masses, momenta, quantum numbers, etc.) of the incoming and outgoing observable particles, and \(\boldsymbol{\alpha}\) are terms which describe the internal structure process represented by the Feynman diagram, including the coupling constants which quantify the interaction strength of the particles involved at each interaction vertex and an integration over the possible dynamics of the virtual particles. 


\section{Evidence for Dark Matter from Observational Astronomy}

Many independent lines of astronomical observation collectively provide compelling evidence for the presence and abundance of a form of matter in the universe that is distinct from the matter that constitutes stars and planets, and which is not directly observable because it neither emits nor absorbs light. Some of the earliest and clearest evidence for this so-called ``dark matter" came in 1978, when Rubin et al. \cite{Rubin_et_al} reported systematic anomalies in their observations of the rotation speeds of spiral galaxies. In particular, it was found that distributions of the rotation speed as a function of the radial distance from the galactic centre differed in shape from what would be naively expected on the basis of the distribution of galactic mass measured from the observed luminosity profile. 

\begin{figure}[H]
	\centering
	\includegraphics[width=0.7\textwidth]{Figures/1/m33_rotation.pdf}
	\caption[]{Observed rotation speed of the nearby dwarf galaxy M33, overlaid on an optical image of the galaxy. Yellow data points show observed rotational speed of the galaxy as a function of the radial distance from the galactic centre (in kpc). Dashed line shows the expected rotational speed on the basis of the calculated mass of the luminous stellar disk. }
	\label{fig:m33_rotation}
\end{figure}

At the time, spiral galaxies had been observed to be comprised of a central spheroidal ``galaxy bulge" which contains the majority of luminous matter in the galaxy, in addition to a ``disk" extending out to larger radii, for which the luminous matter falls off exponentially. Assuming that the distribution of mass in the spiral galaxy follows the luminosity profile, application of Newtonian gravitational mechanics would predict the rotation speed to peak near the edge of the central galaxy bulge, as illustrated in the blue dashed line in Figure \ref{fig:m33_rotation} for the dwarf galaxy M33, and fall off beyond due to the exponentially decaying matter density of the disk. However, the observed galactic rotation speed, shown with yellow data points in Figure \ref{fig:m33_rotation}, is generally observed to continue increasing well beyond the luminous galactic bulge. These anomalies in galactic rotation curves, which have since been observed in hundreds of spiral galaxies \cite{rotn_curves_1995}, can be explained by postulating an additional source of non-luminous matter density in galaxies, known today as ``dark matter", would extend well beyond the luminous bulge, provide the necessary gravitational potential to prevent the rotation speeds from falling off beyond the bulge.

In the years following these early reports of anomalous galactic rotation curves which suggested the existence of dark matter in galaxies, modifications to the laws of Newtonian gravity at galactic scales \cite{mond_1983} were also considered as an alternative to dark matter to explain the observed anomalies. However, while the proposed modifications to gravity were successful in describing the observed galactic rotation curves, numerous lines of astronomical observations in other contexts have independently turned up discrepancies between the observed content of total and luminous matter which would either require further modifications to the laws of gravity, or simply cannot be readily explained by modifying the laws of gravity, thus lending strong confidence hypothesis that a large fraction of the matter density of the universe is non-luminous. Additional evidence at galactic scales comes from strong differences between the spatial distributions of matter density, measured using gravitational lensing, and of luminous matter following collisions of galaxy clusters \cite{bullet_1995}, indicating that the majority of the matter density in the colliding galaxies is non-luminous. Studies of the relative contribution to the masses of galaxy clusters from luminous matter using data from the Chandra X-ray observatory \cite{Chandra_2013} suggest that only 15-20\% of the mass composition of the galaxies studied was comprised of luminous matter.

Because of their stability and EM interactions, protons and neutrons comprised of bound quarks, as well as their bound electrons - collectively known as ``baryonic matter" - comprise by mass the overwhelming majority of known luminous matter in the universe. Precision measurements of the abundance of light nuclei - D, 3He, \(^4\)He, and \(^7\)Li  produced in the early universe following the big bang - known as big bang nucleosynthesis (BBN) \cite{pdg_2018} - indicate that baryonic matter constitutes approximately 5\% \cite{pdg_2018} of the energy density of the universe. Current measurements of anisotropies in the cosmic microwave background \cite{cmb_1965} measured by the Planck collaboration \cite{Planck_2020} interpreted in the context of the standard \(\Lambda\)CDM model of cosmology \cite{pdg_2018} indicate that approximately 30\% of the energy density of the universe is comprised of matter, with the missing 25\% identified as non-baryonic dark matter. This result implies that \(~85\%\) of all matter in the universe is comprised of non-luminous dark matter, consistent with the findings discussed above from measurements of galaxy clusters.

\section{Dark Matter Composition Hypotheses}

There is at this stage a diverse range of astronomical observations which consistently point to the need for a non-luminous form of non-baryonic matter known as dark matter (DM), in the universe. While active research continues within the theoretical community \cite{mond_2012, mond_2021} into the possibility of modifying the laws of gravitation at astronomical scales to explain these observations without the need for DM, there are significant theoretical challenges involved with designing modifications that can consistently explain the range of observational anomalies at scales ranging from individual galaxies to galaxy clusters, while simultaneously addressing the apparent need for DM at cosmological scales from the discrepancy between measurements of the baryonic mass density from BBN and the much larger total mass density inferred from anisotropies in the CMB. As a result, DM is widely considered most theoretically well motivated hypothesis to explain the full range of observational data.

While the astronomical observations provide a wealth of information regarding the composition of DM in the universe by means of its gravitational effects on visible matter, they provide relatively few clues as to what actually comprises the DM. Its abundance in the present day universe indicates that it must be stable on cosmological timescales (i.e. billions of years). The evidence from BBN and CMB anisotropies indicates that the DM must be non-baryonic. Its non-luminous nature further implies that it neither emits nor absorbs photons, and therefore has negligible or no charge under the EM force. Besides baryons, neutrinos represent the only other massive stable particles currently known to the SM. While the masses of individual neutrinos is relatively tiny\footnote{Current constraints from cosmology place an upper limit on the sum of neutrino masses from all generations of 0.17 eV, \(~3\times10^6\) times smaller than the electron mass}, they satisfy the requirement of being electrically neutral, which naturally leads to the question of whether there could be a sufficiently large density of neutrinos in the universe to constitute the observed DM density. This possibility was ruled out in the 1980's by studies \cite{neutrino_dm} which showed that the large scale structure of the universe would differ significantly from what is observed today if the mass density of the universe were dominated by neutrinos due to their ultrarelativistic velocity. More generally, analysis of the measured anisotropies in the CMB \cite{Planck_2020} is found to strongly favour the standard \(\Lambda\)CDM model in which the dark matter content of the universe is predominantly comprised of ``cold" particles which travel at non-relativistic velocities.

With the stable particles of the SM ruled out, the current most widely accepted hypothesis is that DM is comprised of a new form of cold non-baryonic matter that is not currently described by the SM, and which has not yet been observed in particle detectors.

\subsection{Origin and Interactions of Particle Dark Matter}

Despite the observable effects of its gravitational interactions at astronomical scales, the strength of gravitational coupling between massive particles is \(\sim30\) orders of magnitude weaker than any of the other three forces known to the SM \cite{griffiths_2008}. As a result, gravitational interactions between DM and SM particles are far too weak to be observable in particle detectors. Given that there have not yet been any conclusive indications of dark matter in particle detectors, it can be further deduced that any non-gravitational interactions between DM and particles of the SM are relatively weak compared with couplings between SM particles by means of the strong, weak and EM forces known to the SM. However, most theories which aim to describe the origin of the observed abundance of dark matter in the present day universe imply the existence of non-gravitational couplings between DM and SM particles, and in many cases predict strong enough couplings to be probed by modern particle detection methods. 

This produces a generic class of dark matter candidates known as weakly-interacting massive particles (WIMPs), where the ``weak" interactions are not necessarily associated with the weak force, but are simply too weak to have produced a measurable signature in particle detectors to date. The strength of the WIMP hypothesis has inspired worldwide efforts spanning several decades to design increasingly sensitive particle detectors and targeted studies to detect evidence of WIMPs. Such a detection would not only confirm the WIMP hypothesis, but would also allow physicists to begin to study the properties of dark matter as a particle, and perform detailed tests of theoretical extensions to the SM which incorporate dark matter.

\subsubsection{Dark Matter Origin from Thermal Freeze-out}

A review of the existing hypotheses for the origin of dark matter can be found in Section 26.3 of Ref. \cite{pdg_2018}. Of these, the so-called ``thermal freeze-out" scenario is a popular candidate, because it postulates that the observed dark matter density in the present day universe originated from the same process of thermal decoupling that produced the primordial abundances of light nuclei in the well-supported BBN scenario. The hypothesis postulates that in the very early universe, matter was sufficiently dense and energetic to establish thermal equilibrium between DM and SM particles due DM-SM interactions. As the universe expanded and cooled, eventually the rate of DM-SM interactions became too low to maintain thermal equilibrium between the two species. At this point, the DM-SM interaction rate became too low to appreciably modify the relative density of DM and SM particles, thus producing the relic abundance of dark matter observed in the present-day universe. 

For cold DM relics (\(v/c\lesssim0.1\) at the time of freeze-out), and assuming that the relic abundance is predominantly set by direct DM-SM interactions, analysis of the observed relic abundance of DM in the context of the thermal freeze-out hypothesis \cite{dm_xsec_2015} implies that the cross section for SM-DM interactions should be \(\sigma_\text{SM-DM}\gtrsim1\) pb, comparable typical interaction strengths mediated by the weak force. While searches for dark matter in particle detectors have yet to turn up any hints of a dark matter candidate with interaction cross sections with the SM near the weak scale \cite{wimp_searches_2018}, the cross section constraint can be significantly relaxed by considering a scenario in which the relic abundance of dark matter is set not by direct interactions between the DM and the SM, but rather by interactions between DM and an unstable mediator, which subsequently decays to SM particles \cite{secluded_dm_2008}. The DM search presented in this thesis is interpreted in the context of such a scenario, wherein the unstable mediator is a scalar boson referred to as the ``dark Higgs boson" \cite{Duerr_2016,Duerr2017}.

\section{Dark Matter Search Strategies}

There are three complementary approaches used to search for particle DM by means of its non-gravitational interactions: direct and indirect detection, and collider searches. Direct detection searches \cite{Schumann_2019, 2015gya, billard2021direct} aim to directly detect evidence of a recoil induced by elastic scattering between a DM particle in the galactic halo passing through the detector and a target particle in the detector. Indirect searches \cite{CIRELLI_2012, conrad} use observational data to search for evidence of products produced by DM annihilation or decay in particular regions of the observable universe expected to have a high DM density. Collider searches \cite{DM_colliders}, of which the proposed thesis work is an example, study the decay products from high-energy collisions of subatomic particles to search for an above-background excess of events that could be consistent with DM having been produced in some of the collisions.

\subsection{Direct Detection}

Direct detection searches operate in very low-background environments, typically in underground facilities such as SNOLAB \cite{Lawson_2020}, in order to minimize elastic scattering events in the detectors from non-DM sources such as cosmic rays and radioactivity, and detailed studies are performed to determine the expected rate of events from all such background sources. As a result, a significant excess of elastic scattering events, particularly if observed in multiple direct detection experiments, would offer a clear signature of DM in the galactic halo. 

If no evidence of excess scattering events is found, experiments place upper bounds on DM-nucleon interaction cross section with a largely standard set of methods and assumptions (most notably the local DM density and the relative speed with which the DM passes through Earth) \cite{dd_results_standards_2021}, which facilitates comparison between different experiments. Figure \ref{dd_limits} summarizes the current upper bounds on the spin-independent\footnote{Spin-dependent vs. spin-independent DM-nucleon differ according to whether the coupling is sensitive to the spin state of the target nucleon \cite{billard2021direct}.} DM-nucleon interaction cross section from all direct detection searches. The searches probe down to many orders of magnitude below the weak scale (\(\sigma\sim10^{-36}\)cm\(^{-2}\)) over \(\sim4\) orders of magnitude of candidate DM masses. However, current direct detection strategies generally suffer practical limitations to the ranges of DM masses and interaction cross sections that can be probed. The lower bound on accessible DM masses is in general dictated by the signal to noise ratio of the detector, referred to as the ``noise wall", which is quite difficult to overcome. The range of accessible cross sections is also bounded from below for most direct detection experiments by the so-called ``solar neutrino floor", below which the measured event rate becomes dominated by the irreducible flux of solar neutrinos through the Earth. 

\begin{figure}[h]
	\centering
	\includegraphics[width=0.7\textwidth]{Figures/1/dd_results.pdf}
	\caption[]{Summary of upper bounds on the interaction cross section for spin-independent WIMP-nucleon scattering from direct detection searches. Figure from \(\copyright\) \cite{billard2021direct}.}
	\label{fig:dd_limits}
\end{figure}

\subsection{Indirect Detection}

By searching for excesses of several potential DM annihilation products in observational data - gamma rays and charged leptons and antimatter - in addition to neutrinos \cite{conrad2014indirect, pdg_2018}, indirect detection searches can avoid the limitation of the solar neutrino floor. Depending on the target species, these searches can also target a wider range of candidate DM masses compared with direct detection searches \cite{pdg_2018}. Due to the many potential processes that could produce the target particles in observational data - both within and beyond the SM - indirect searches generally contend with relatively large uncertainties associated with modelling the expected flux from these background sources. Indirect searches generally probe \(\langle \sigma v \rangle\), where \(\sigma\) is the the DM-DM annihilation cross section, \(v\) is the velocity of the target DM relative to Earth, and the average represented by \(\langle\rangle\) is over the expected distribution of \(v\) as a function of the DM mass, for an assumed dark matter density in the target region. 

\subsection{Collider Searches}

Rather than searching for non-gravitational interactions of relic DM on earth or in the observable universe, collider searches \cite{DM_colliders} instead search for evidence of dark matter production from SM-SM interactions in high-energy particle collisions. Like neutrinos, DM would be expected to pass invisibly through any detector surrounding the collision point due to its very low interaction cross section, producing a momentum imbalance transverse to the beam line\footnote{See Section \ref{sec:met} for a detailed introduction to missing transverse momentum in the ATLAS detector.}. An excess of collision events with a final-state momentum imbalance over the rate expected for SM processes which produce neutrinos would be a strong indication of the production of a massive, stable weakly-interacting particle, consistent with the expected properties of DM. Given that other hypothetical new physics processes \cite{add_1998,dark_energy_lhc} could produce a momentum imbalance from a non-DM source, any such excess seen in colliders would be strengthened by corroborating DM detections in direct and indirect detection experiments. 

Despite operating in a very high background environment, collider searches offer numerous advantages which allow them to complement and potentially extend the reach of direct and indirect searches. First, detectors surrounding collision points at particle colliders are typically designed to measure the final-state particles produced by the collisions and their kinematic information with high precision in order to perform a range of measurements. The detailed final-state information allows DM searches to target specific final-state topologies, which can lead to substantial reductions in SM background processes and considerably enhance the sensitivity to hypothetical DM production processes which predict collision events with the targeted topology. 

Second, by targeting DM produced in the collisions and adopting a search strategy (missing momentum) that does not require the DM to interact with the detector, collider searches are insensitive to the neutrino floor that will challenge the sensitivity of next-generation direct detection searches. 

Third, while the range of DM masses is bound from above by the centre of maximum centre of mass energy of the particle collisions (\(\sim TeV\) for proton-proton collisions at the LHC), collider searches do not suffer the so-called noise  wall that limits the sensitivity of direct detection searches below \(\sim1~\GeV\) (see Figure \ref{dd_results.pdf}).

Regardless of where particle DM may be first discovered, the detailed final-state information available in particle collision data means that colliders will be an indispensable tool for studying the properties and interactions of the new particle(s).

\subsection{Searching for Dark Matter at Particle Accelerators}

\subsubsection{Collider Search Approaches}

The concept of searching for evidence of DM production in high-energy particle collisions is currently being exploited in a variety of collider experiments. 

\subsubsection{Models of DM Production}

Models of DM production in colliders can range in complexity from an effective field theory (EFT), where the DM production mechanism is completely unspecified, to a complete model such as supersymmetry \cite{susy_dm}, which predicts viable DM candidates as part of a hypothesized extension to the SM designed to address a range of phenomena unexplained by the SM. 

The EFT approach treats the production of DM from colliding partons as a contact interaction, with the production rate determined by a single parameter \cite{DM_colliders}. As long as a measurable SM particle is also produced in the interaction (eg. a gluon radiating from one of the colliding quarks, see figure \ref{fig:eft_simplified_model}), the EFT framework can be applied to any mono-X signature at the LHC, where a SM particle X is measured along with missing transverse momentum in the detector. This makes the framework generally usable in terms of motivating and providing a theoretical framework to interpret a range of generic search channels that can be readily selected for in LHC collision data. However, the EFT framework relies on the assumption that the the mediator(s) of the interaction is (are) much more massive than the scale of momentum transfer in the interaction \cite{DM_colliders, beyond_eft}. If this assumption is inaccurate, the EFT framework becomes invalid, and a more complete model is needed to specify additional details of the process leading to DM pair production. 

\begin{figure}[H]
	\centering
	\begin{minipage}[b]{0.45\textwidth}
	\includegraphics[width=0.9\textwidth]{Figures/1/EFT_Signature.png}
	\end{minipage}
	\begin{minipage}[b]{0.45\textwidth}
	\includegraphics[width=0.8\textwidth]{Figures/1/simplified_model.png}
	\end{minipage}
	\caption[]{Left: Mono-jet process in the EFT framework (source: \cite{beyond_eft}). Right: Mono-jet process in a simplified model framework, where the pair production of DM occurs via a new vector or axial-vector ($V$, $A$) mediator of mass $M_\text{med}$, which couples to quarks and DM with coupling constants g$_q$ and g$_\text{DM}$, respectively (source: \cite{dm_forum})}.
	\label{fig:eft_simplified_model}
\end{figure}

In principle, complete theories of physics beyond the SM, such as the minimal supersymmetric SM (MSSM) \cite{mssm} can offer theoretically motivated and experimentally accessible models which specify the details of candidate processes by which the colliding partons may annihilate to produce DM. However, these theories tend to be quite complex, with many free parameters - over 100 in the case of MSSM \cite{DM_colliders} - most of which need to be fixed to generate a reasonably testable model. Relying on complete theories alone to guide experimental signatures may run the risk of missing important parameter space of new physics for which a complete theory has not yet been developed. 

Simplified models, widely used in recent and ongoing DM searches at the LHC, are designed to bridge the gap between EFT and complete theories. They provide a `first-order' description of theoretically motivated new physics scenarios that could be accessible at collider energies. They provide guidance for experimental searches without fully specifying the details of any additional new physics at energies above the collider scale that would be needed for a complete theory \cite{DM_colliders}. In terms of DM production at the LHC, one or more new mediators associated with new physics scenarios may be considered which allow for mixing between SM particles and DM. The process by which the mixing occurs is represented with a tree-level diagram whose experimental signature would be accessible at LHC energies, such as the diagram shown in figure \ref{fig:eft_simplified_model} which represents a DM benchmark model featured in the 2015 report of the ATLAS/CMS Dark Matter Forum \cite{dm_forum}.

\begin{itemize}
\item Give an idea of the breadth of accelerator DM search program.
\begin{itemize}
\item Resonance searches
\item mono-X searches
\item Model-independent and model-dependent approaches $\rightarrow$ briefly present range of model completeness, from EFT through simplified to complete.
\item Cover searches at hadron and $e^+e^-$ colliders and fixed target experiments.
\end{itemize}
\item Discuss the relative merits of searching for simplified models (bridge gap between EFT and complete theories, avoid overtuning problems inherent with complete models, facilitate adequate coverage of plausible DM production processes).
\item Briefly introduce the Dark Higgs model, and clearly identify it as a simplified model (details to be fleshed out in Chapter 2) $\rightarrow$ emphasize that this search is sensitive to heavy DM ($\gtrsim$60 GeV), with the requirement that $m_\chi>\frac{1}{2}\ms$ (to prevent the $s\rightarrow\chi\chi$ decay mode).
\end{itemize}

