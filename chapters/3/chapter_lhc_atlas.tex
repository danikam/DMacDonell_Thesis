\startchapter{Introduction to the LHC and the ATLAS Detector}
\label{chapter:lhc_atlas}

The Large Hadron Collider (LHC) \cite{lhc_machine} is a circular proton-proton collider which resides in a 27 km tunnel near the European Organization for Nuclear Research (CERN). Superconducting magnets are used to accelerate counter-rotating bunched proton beams to near the speed of light, and direct the beams into head-on collisions at four interaction points around the ring. The collisions take place at a world-leading centre of mass energy of up to 13 TeV. 

The large centre of mass energy of the collisions makes it possible for the colliding proton constituents, known as ``partons", to annihilate and produce massive unstable particles such as the Higgs boson, which cannot presently be produced by any other experimental means. By studying the decay products of massive particles produced at the LHC, experiments such as ATLAS can study hypothesized DM production mechanisms which would proceed via the decay of these massive particles. 

Each interaction point is surrounded by a detector, which measures the energetic debris of particles produced by the high energy collisions to perform precision measurements of the SM and search for new physics. ATLAS (A Toroidal LHC ApparatuS) \cite{atlas} is one of two multi-purpose detectors at the LHC, designed to record and study a wide range of physics processes resulting from the collisions. 

\section{Introduction to the LHC}
\begin{itemize}
\item Location, circumference, CoM energy, luminosity. Try to give some intuitive meaning to the CoM energy and luminosity. Emphasize that the large CoM collision energy makes it possible to probe models of new physics with heavy mediators, and the high luminosity is needed to produce sufficient statistics to search for statistically significant discrepancies between data and MC which could be indicative of new physics. 
\item Four interaction points, each surrounded by a detector $\rightarrow$ give a brief introduction (~1 sentence) of the other three detectors to give an idea of how ATLAS fits into the wider LHC physics programme.
\item Brief discussion of the parton model and PDFs, making the point that it's the partons colliding at the LHC, and that the colliding partons and their fraction of momentum carried are probabilistic and described by the proton PDF for a given momentum transfer scale. This will be useful both for the discussion of \met, and in case I want to mention our study of the potential benefit of binning in lepton charge.
\end{itemize}

\section{Introduction to the ATLAS detector}

The ATLAS detector, shown schematically in figure \ref{fig:detector}, provides full 4$\pi$ coverage around the interaction point, with the exception of the beam pipe. It consists of several layers of sub-detectors, each of which is specialized for recording certain kinematic information and particle types. The sub-detectors are described in some detail below. 

\begin{itemize}
\item Introduction to the ATLAS detector, giving an idea of its scale and significance as one of the two general purpose particle detectors at the LHC (enables a wide range of physics measurement and search programmes; complementarity with CMS).
\item Inner detector $\rightarrow$ discussion of charged particle tracking will be relevant for later description of TAR jet reconstruction (may want to point that out already).
\item Calorimeters $\rightarrow$ emphasize distinction between small- and large-radius jets, and between electromagnetic and hadronic showers. 
\begin{itemize}
\item Talk about electron detection after/during the description of the EM calorimeter (should have all needed info since the inner tracker has already been discussed). 
\end{itemize}
\item Muon spectrometer for muon detection $\rightarrow$ emphasize that muons pass through the other sub-detectors. 
\item \met 
\begin{itemize}
\item Define \met here, now that all sub-detectors have been described. This intro to \met will be needed for the discussion of the \met trigger in the next section.
\item Shouldn't need to go into too much detail on the objects involved in \met reconstruction, since this will be covered in more detail in Chapter 5.
\end{itemize}
\item Trigger system 
\begin{itemize}
\item Discuss the relatively enormous cross sections of soft QCD processes $\rightarrow$ emphasize that much of the trigger design and data selections are devoted to reducing this soft QCD background to focus on the rarer physics processes of interest for measurements/searches. 
\item Otherwise, focus on the \met trigger, and mention that it only uses info from the calorimeter (relevant for later presentation of our use of the \met OR single muon trigger).
\end{itemize}
\end{itemize}
