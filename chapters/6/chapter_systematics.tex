\startchapter{Systematic Uncertainties}
\label{chapter:systematics}

In order to properly assess the significance of any discrepancies that may be seen between predicted yields and distributions of signal and SM background and the observed collision data, it is important to assign an uncertainties to all sources which may limit the precision and accuracy of the yield predictions. In addition to the limited precision of the estimates due to statistical uncertainty\footnote{See Chapter \ref{sec:mc_intro} for a discussion of the origin of statistical uncertainty in MC simulations.}, there may be inaccuracies in the values of various parameters involved in the simulation due to a limited precision with which their values are known. These inaccuracies can systematically shift predicted production rates and kinematic properties of the simulated processes, or the objects they produce in the ATLAS detector, and ultimately predicted yields and shapes of the processes in the analysis regions. Systematic uncertainties aim to quantify the maximal range of yield shifts, in each region and bin\footnote{See Section zzz \textcolor{red}{(Note to Bob: will update when section on binning is written)} for details of the binning in \minms performed in the SR} used in the search, that could result from a potential source of systematic inaccuracy. 

Systematic uncertainties (or simply ``systematics") are classified into two main categories: theoretical and experimental. Theoretical systematics account for inaccuracies that could result from a limited knowledge of parameters involved in modelling the production and decay mechanisms resulting from \(pp\) collision events at the LHC. Experimental systematics account for inaccuracies involved in the physics objects\footnote{See Section \ref{sec:object_defs} for a presentation of the reconstructed physics objects used in this search.} used to reconstruct the decay products of collision events in the ATLAS detector, and in the LHC machine itself. 

Experimental systematics are generally constrained in the process of calibrating particular physics objects reconstructed in the ATLAS detector using the ATLAS collision data, and as a result are generally considered quite robust. In contrast, the choice of an uncertainty to assign to theoretical parameters can be much less clear, as constraints from experimental results may be sparse, or in some cases such as appropriate choice of renormalization scale in perturbative QCD calculations \cite{PDG_2018}, essentially non-existent. 

\subsection{Experimental Systematics}

Experimental systematics are evaluated for all physics objects considered in the analysis, and for the LHC beam luminosity. 

The integrated luminosity \(\mathcal{L}_\text{int}\) recorded by the ATLAS detector for the full data set considered in this search is known with a precision of \(1.7\%\) \cite{ATLAS-CONF-2019-021}. Since, from Eq. \ref{eq:integrated_lumi}, the total number of recorded collision events scales linearly with the integrated beam luminosity, propagating this \(\pm1.7\%\) systematic to the yields results in coherent \(1.7\%\) up and down shifts in all bins of the predicted yields for all processes.

For a given systematic uncertainty on a parameter \(k\) in physics object, the general procedure for propagating the systematic uncertainty to the predicted yield \(N\) in a bin \(j\) is as follows:

\begin{itemize}
\item Repeat the reconstruction with \(p\) shifted up by one standard deviation: \(k\rightarrow k+\sigma_k\).
\item Evaluate the predicted yield in the bin \(N_\text{\(k\), up, bin \(j\)}\) with updated simulation. The ``up" systematic yield uncertainty is \(\text{syst}\text{( \(k\), up, bin \(j)\)} = N_\text{\(k\), up, bin \(j\)} - N_\text{nom, bin \(j\)}\), where \(N_\text{nom, bin \(j\)}\) is the nominal yield with the unshifted \(k\).
\item Repeat the above process with \(k\) shifted down by one standard deviation to evaluate the ``down" systematic uncertainty.
\end{itemize}

\noindent Due to the statistical uncertainties arising from the limited number of MC events available to propagate systematic shifts to associated yield shifts, asymmetries between the up and down yield shifts from the above procedure can occur simply due to statistical fluctuations in the number of events which fall into each bin, rather than from actual asymmetries in the underlying distribution of shifted yields. For this reason, the propagated yield systematics are symmetrized in each bin as follows:

\begin{equation}
\label{eq:exp_systs_symm}
\text{syst}\text{(exp \(k\), symm, bin \(j\))}= \pm\Bigg(\frac{N_\text{\(k\), up, bin \(j\)} - N_\text{\(k\), down, bin \(j\)}}{2}\Bigg)
\end{equation}

Table \ref{zzz} summarizes the uncertainties considered for all physics objects in the search. 



\subsection{Theoretical Systematics}


\begin{itemize}
\item Discussion of each major source of experimental systematics, how they're evaluated and some plots and tables showing shifts in yield arising from the dominant sources of experimental systematics.
\item Discussion of the sources of theory systematics considered (QCD scale, PDF, $\alpha_s$, PS) and how each is evaluated for signal and background.
\item Plots and tables showing the yield shifts and relative sizes of each source of theory uncertainty for signal and major backgrounds.
\item Comparison of the relative impact of statistical vs. systematic uncertainties in each analysis region, to identify where we're stats- vs. systematics-dominated.
\end{itemize}