\startchapter{Systematic Uncertainties}
\label{chapter:systematics}

\begin{itemize}
\item General discussion of how systematic uncertainties can bias the simulation of physical objects, ultimately resulting in a systematic mis-modelling of the yields and shapes of simulated processes in our analysis regions.
\item Discussion of how systematic uncertainties are evaluated using the up/down shift in the yield within each bin obtained by varying the given source of systematic uncertainty.  
\item Clearly distinguish between theoretical and experimental sources of systematic uncertainty. Discuss the relative difficulty often involved with assigning reliable theoretical uncertainties (can be challenging to assess realistic bounds on the possible range of values for theoretical inputs, whereas experimental bounds are generally more measurable). 
\item Discussion of each major source of experimental systematics, how they're evaluated and some plots and tables showing shifts in yield arising from the dominant sources of experimental systematics.
\item Discussion of the sources of theory systematics considered (QCD scale, PDF, $\alpha_s$, PS) and how each is evaluated for signal and background.
\item Explanation of the special treatment of the triboson theory uncertainty (i.e. setting it to 100\%) due to the absence of any appropriate alternate generator sample.
\item Plots and tables showing the yield shifts and relative sizes of each source of theory uncertainty for signal and major backgrounds.
\item Comparison of the relative impact of statistical vs. systematic uncertainties in each analysis region, to identify where we're stats- vs. systematics-dominated.
\end{itemize}