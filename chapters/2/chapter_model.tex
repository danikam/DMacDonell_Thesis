\startchapter{The Dark Higgs Model}
\label{chapter:dh_model}

\section{Theoretical Motivation for the Dark Higgs Model}
\begin{itemize}
\item Dark Higgs needed to generate masses of dark sector particles.
\item Need for creation and annihilation mechanism between SM and DM in early universe (thermal freeze-out hypothesis) motivates mixing between SM Higgs and dark sector Higgs.
\end{itemize}

\section{Model Description}
\begin{itemize}
\item Introduce it in the context of the wider class of dark sector models.
\item Describe the production mechanism, including leading Feynman diagrams.
\item Specify couplings to both SM and dark sector particles. 
\item Emphasize that $m_\chi>\frac{1}{2}\ms$ required to obtain signature in detector (otherwise $s\rightarrow\chi\chi$ decay would dominate).
\end{itemize}

\section{Search for the Dark Higgs Model at the LHC}

\begin{itemize}
\item Discussion of the model's signature in the ATLAS detector (boosted SM pair recoiling against \met), and why the boosted topology is unique compared with generic mono-X searches in which the `X' is produced via ISR.
\item Discussion of available search channels - mono-s(bb), mono-s(WW), mono-s(ZZ), mono-s(hh).
\begin{itemize}
\item Overview of existing searches for the Dark Higgs model - re-interpreted mono-h(bb), hadronic mono-s(WW). Could also mention ongoing dedicated mono-s(bb) search.
\item Identify the \ms regime in which the $s\rightarrow WW$ decay mode dominates in sensitivity.
\end{itemize}
\end{itemize}