\startchapter{The Dark Higgs Model}
\label{chapter:dh_model}

The dark matter (DM) search presented in this thesis is motivated by and interpreted with the Dark Higgs (DH) model \cite{Duerr2017}. The DH model predicts a mechanism for DM production from proton-proton collisions at the LHC by means of portal interactions with the dark sector. The dark sector, which is predicted as part of various BSM physics models, represents a collection of quantum fields and associated particles which are assumed to interact gravitationally, but which do not couple via any of the other known forces - electromagnetic, strong and weak - of the SM. Non-gravitational couplings between the darks sector and the SM proceed instead via one or more so-called ``portal mediators". 

In the DH model, the DM is a particle which belongs to the dark sector, and a measurable signature of DM is . The model  from \(q\bar{q}\) collisions at the LHC via the exchange of a new spin 1 gauge boson referred to as the \Zprime boson. To 

is produced from \(q\bar{q}\) collisions via a hypothetical spin 1 portal mediator referred to as the \Z' boson, which decays to a pair of DM particles. A measurable SM signature is produced by means of the emission of an additional. Figure \ref{} shows three Feynman diagrams which illustrate some of the dominant modes by which a measurable  

\section{Theoretical Motivation for the Dark Higgs Model}

Given that the particles of the SM acquire mass via their interaction with the Higgs field,  the hypothetical ``Dark Higgs" field - and its associated particle the Dark Higgs (DH) boson is motivated by the need to likewise generate masses of particles in the dark sector.

\begin{itemize}
\item Dark Higgs needed to generate masses of dark sector particles.
\item Need for creation and annihilation mechanism between SM and DM in early universe (thermal freeze-out hypothesis) motivates mixing between SM Higgs and dark sector Higgs.
\end{itemize}

\section{Model Description}
\begin{itemize}
\item Introduce it in the context of the wider class of dark sector models.
\item Describe the production mechanism, including leading Feynman diagrams.
\item Specify couplings to both SM and dark sector particles. 
\item Emphasize that $m_\chi>\frac{1}{2}\ms$ required to obtain signature in detector (otherwise $s\rightarrow\chi\chi$ decay would dominate).
\end{itemize}

\section{Search for the Dark Higgs Model at the LHC}

\begin{itemize}
\item Discussion of the model's signature in the ATLAS detector (boosted SM pair recoiling against \met), and why the boosted topology is unique compared with generic mono-X searches in which the `X' is produced via ISR.
\item Discussion of available search channels - mono-s(bb), mono-s(WW), mono-s(ZZ), mono-s(hh).
\begin{itemize}
\item Overview of existing searches for the Dark Higgs model - re-interpreted mono-h(bb), hadronic mono-s(WW). Could also mention ongoing dedicated mono-s(bb) search.
\item Identify the \ms regime in which the $s\rightarrow WW$ decay mode dominates in sensitivity.
\end{itemize}
\end{itemize}