\startchapter{Object Definitions, Triggers and Event Selection}
\label{chapter:objects}

This chapter describes the physics objects which are reconstructed on an event-by-event basis using collision data from the ATLAS detector and used in this DM search. It also discusses the triggers and event selection cuts which are applied to define the subsets of collision data and MC simulated data, also known as ``analysis regions", which are used for the search. 

\section{Object Definitions}
\label{ap:object_defs}

The goal of the ATLAS detector is to identify particles that are produced by the proton-proton collisions that take place in the centre of the detector, and to reconstruct their kinematic properties. The particle identification and reconstruction is performed using various collections of measured signals in the detector sub-systems, which are broadly referred to as ``physics objects" \cite{physics_objects_atlas_2013}. The physics objects used to reconstruct all particles considered in this search are described in the following sections.

\subsection{Charged Leptons}
\subsubsection{Electrons}
\subsubsection{Muons}
\subsection{Small-radius \aktfour jets}
\subsection{TAR jets}
\subsection{\met}
\subsection{Dark Higgs Candidate Mass}

\section{Triggers}
\label{sec:triggers_evt_selection}

As described in Section \ref{sec:trigger}, the ATLAS trigger system only saves collision events during data collection which pass both the hardware-based level-1 (L1) trigger and the software-based high-level trigger (HLT). The L1 trigger and the HLT are each comprised of numerous sets of selection criteria, which are also referred to as triggers. Any collision event which satisfies at least one of the triggers which comprise the L1 trigger is processed by the HLT, and likewise if the event satisfies any of the triggers which comprise the HLT it will be saved to storage.

As part of the selection of events to be considered in an analysis, events are typically required to have passed some combination of HLT triggers which are relevant for the analysis. The search presented in this thesis is interested in events which produce a single energetic lepton due to the \(s\rightarrow WW(q\bar{q}\ell\nu)\) decay, in addition to high \met due to both the undetected boosted DM in the final state and the undetected \(\nu\) from the \(W\rightarrow \ell\nu\) decay. Therefore, in order to be considered for this search, events are required to have passed either the high-\met trigger (also referred to simply as the ``\met trigger), or the single muon trigger. 

\begin{itemize}
\item State the trigger combination used for the search (\met OR single muon)
\item Show trigger efficiency curves for \met only and \met OR single muon to show that the (OR single muon) is a necessary addition in the muon channel to achieve 100\% sensitivity.
\item Explain why \met trigger alone is insufficient in the muon channel due the exclusive use of calorimeter information by the ATLAS \met trigger.
\end{itemize}

\section{Event Selections}
\label{sec:evt_selections}

\begin{itemize}
\item High level discussion of why we apply event selections, and goals for optimal signal region definition.
\begin{itemize}
\item Broadly: maximize predicted signal content and minimize simulated background content, while maintaining sufficient MC and data statistics to enable a meaningful comparison between MC and data.
\item Prioritize optimization of signal points near the edge of expected search sensitivity. 
\item Keep signal region blind during optimization to avoid biasing selection.
\end{itemize}
\item Introduce variables used for event selection. Distinguish between variables that are optimized (eg. \mtlepmet) vs. fixed (eg. 1-lepton requirement) during optimization.
\item Present concept and implementation of signal region optimization strategy.
\item High-level discussion of why we define CRs to constrain normalizations of dominant \wjets and \ttbar backgrounds.
\begin{itemize}
\item Provides data-driven normalization constraint which can be extrapolated to the signal region (more details on extrapolation procedure in Chapter 7)
\item Reduces the impact of (and reliance on) theoretical uncertainties involved in simulating the correct normalizations for these backgrounds. Emphasize the difficulty involved with assigning reliable theoretical uncertainties, and hence the value of using data-driven constraints.
\end{itemize}
\item Summary of design goals for control region
\begin{itemize}
\item High purity of background of interest.
\item Orthogonal to SR.
\item Phase space kinematically similar to SR.
\item Signal contamination negligible compared with uncertainty of total background yield.
\end{itemize}
\item Present the \wjets control region, and motivate the \dR reversal used to define it.
\item Present the \ttbar control region, and motivate the \bjet veto reversal used to define it.
\item Present the additional modifications that were needed in the \merged category to optimize the CR definitions
\begin{itemize}
\item Reducing the lower bound on \metsig to boost stats.
\item Increasing the lower bound on \dR in the \wjets CR to reduce the signal contamination to an acceptable level.
\end{itemize}
\item Summary of all analysis regions.
\end{itemize}
