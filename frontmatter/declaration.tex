\begin{center}
DECLARATION
\end{center}

\section*{Analysis contributions}

\begin{itemize}
\item Initiated and served as analysis co-contact for the mono-s(WW) semileptonic analysis presented in this thesis (2020-present).
\begin{small}
\begin{itemize}
\item Defined, delegated, documented and ensured timely completion of studies and tasks for the analysis.
\item Liaised with conveners, experts and performance groups to ensure that analysis choices were widely understood and endorsed.
\item Organized and facilitated regular analysis group meetings.
\item Delivered regular updates and presentations to communicate the progress of the analysis to ATLAS collaborators. 
\item Facilitated and contributed to all stages of analysis approval within the ATLAS collaboration by presenting at approval meetings, interfacing with conveners, and addressing feedback from reviewers.
\end{itemize}
\end{small}
\item Editor and primary developer of the supporting documentation for the publication of the mono-s(WW) semileptonic analysis.
\item Primary analyst for the mono-s(WW) semileptonic analysis.
\begin{itemize}
\begin{small}
\item Collaborated on the design and optimization of selections used to define signal and control regions for the analysis.
\item Optimized and implemented the binning strategy in the signal region.
\item Developed a framework to evaluate and implement all systematic uncertainties considered in the analysis.
\item Designed and optimized the statistical interpretation of the analysis within the HistFitter \cite{Baak_2015} framework.
\item Contributed to the development and release of ATLAS-wide recommendations for the calibration of small-radius (\(R=0.2\)) jets, and for the evaluation of their associated systematic uncertainties. \(R=0.2\) jets are used to construct large-radius TAR jets \cite{TAR_algo} used in the analysis (see Section \ref{sec:TAR_jets} for details).
\item Prepared requests for Monte Carlo samples used by the analysis to model signal and Standard Model background processes, and processed all data and Monte Carlo samples through the full analysis chain.
\end{small}
\end{itemize}
\item Contributed to control region studies and the evaluation of theoretical systematic uncertainties for the mono-s(WW) analysis in the fully hadronic channel.
\end{itemize}

\section*{Computing infrastructure contributions}
\begin{itemize}
\item Developed infrastructure to automate the creation and deployment of a kubernetes cluster as an ATLAS computing site using cloud computing infrastructure at the University of Victoria.
\item Implemented and tested tools for cluster federation and resource monitoring. 
\end{itemize}

\section*{Analysis preservation contributions}
\begin{itemize}
\item Analysis preservation contact person for all searches for new physics within the ATLAS collaboration (Feb. 2020 to Oct. 2021).
\begin{itemize}
\begin{small}
\item Provided technical assistance and liaison to support analysis teams with the development of automated analysis preservation workflows in the RECAST framework \cite{Cranmer2011}.
\item Maintained central documentation of RECAST tools.
\item Reviewed RECAST workflows for completeness, and ensured centralized storage of workflows.
\item Organized hands-on training events to familiarize analysts with the tools involved with analysis preservation.
\item Facilitated the integration of the REANA data analysis platform \cite{reana_2021} as a central tool for the development and execution of RECAST analysis preservation workflows.
\end{small}
\end{itemize}
\item Contributed to workshop planning and ran tutorials on analysis preservation and related tools for numerous ATLAS workshops.
\item Developed and currently maintain the RECAST framework for the ATLAS \met+jets search published in 2021 \cite{monojet_atlas_2021}, which continues to be used regularly within the collaboration to constrain new models of physics beyond the Standard Model.
\item Developed the RECAST frameworks for the mono-s(WW) analysis in both the fully hadronic and semileptonic channels.
\end{itemize}