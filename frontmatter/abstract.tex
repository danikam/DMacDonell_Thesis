\newpage
\TOCadd{Abstract}

\noindent \textbf{Supervisory Committee}
\tpbreak
\panel %adds comittee

\newpage

\begin{center}
\textbf{ABSTRACT}
\end{center}

This dissertation presents a search for dark matter production at the Large Hadron Collider using 139 fb\(^{-1}\) of proton-proton collisions at a centre-of-mass energy of \(\sqrt{s} = 13\,\)TeV, recorded with the ATLAS detector from 2015 to 2018. The search targets a final state topology in which dark matter is produced from the proton-proton collisions in association with a pair of W bosons, one of which decays to a pair of quarks and the other to a lepton-neutrino pair. The dark matter is expected to pass through the detector without detection, resulting in an imbalance of momentum in the plane transverse to the beam line. 
%Candidate events are selected on the basis of missing transverse momentum recoiling against a boosted system of jets and one charged lepton. Contributions to the selected events from Standard Model processes are evaluated using Monte Carlo simulation, and data-driven methods are used to constrain the contributions from dominant \(W(\ell\nu)\)+jets and \(\ttbar(WbWb)\) processes. 
The search is optimized to test the Dark Higgs model, which predicts a signature of dark matter production in association with the emission of a hypothesized new particle referred to as the Dark Higgs boson. The Dark Higgs boson would decay to a W boson pair  via a small mixing with the Standard Model Higgs boson discovered in 2012. Events that exhibit the targeted final state topology are selected for the search, and an approximate mass of the hypothetical Dark Higgs boson is reconstructed for each event. The final search is performed by looking for a deviation between distributions of the reconstructed Dark Higgs boson masses and Standard Model predictions for the selected events. No such deviation is found, so the range of Dark Higgs model parameters probed by the search is excluded. This search complements and extends the reach of existing searches for the Dark Higgs model by the ATLAS and CMS collaborations.



