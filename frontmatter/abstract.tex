\newpage
\TOCadd{Abstract}

\noindent \textbf{Supervisory Committee}
\tpbreak
\panel %adds comittee

\newpage

\begin{center}
\textbf{ABSTRACT}
\end{center}

Longstanding evidence from observational astronomy indicates that non-luminous ``dark matter" constitutes the majority of all matter in the universe, yet this mysterious form of matter continues to elude experimental detection. This dissertation presents a search for dark matter at the Large Hadron Collider using 139 fb\(^{-1}\) of proton-proton collision data at a centre-of-mass energy of \(\sqrt{s} = 13\,\)TeV, recorded with the ATLAS detector from 2015 to 2018. The search targets a final state topology in which dark matter is produced from the proton-proton collisions in association with a pair of W bosons, one of which decays to a pair of quarks and the other to a lepton-neutrino pair. The dark matter is expected to pass invisibly through the detector, resulting in an imbalance of momentum in the plane transverse to the beam line. The search is optimized to test the Dark Higgs model, which predicts a signature of dark matter production in association with the emission of a hypothesized new particle referred to as the Dark Higgs boson. The Dark Higgs boson is predicted to decay to a W boson pair  via a small mixing with the Standard Model Higgs boson discovered in 2012. Collisions that exhibit the targeted final state topology are selected for the search, and an approximate mass of the hypothetical Dark Higgs boson is reconstructed from the particles in each collision. A search is performed by looking for a deviation between distributions of the reconstructed Dark Higgs boson masses and Standard Model predictions for the selected collisions. The data is found to be consistent with the Standard Model prediction, and the results are used to constrain the parameters of the Dark Higgs model. This search complements and extends the reach of existing searches for the Dark Higgs model by the ATLAS and CMS collaborations.



