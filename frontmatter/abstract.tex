\newpage
\TOCadd{Abstract}

\noindent \textbf{Supervisory Committee}
\tpbreak
\panel %adds comittee

\newpage

\begin{center}
\textbf{ABSTRACT}
\end{center}

This dissertation presents a search for dark matter production at the Large Hadron Collider using 139 fb\(^{-1}\) of proton-proton collisions at a centre-of-mass energy of \(\sqrt{s} = 13\,\)TeV, recorded with the ATLAS detector from 2015 to 2018. The search targets a final state topology in which the dark matter is produced in association with a boosted \(W^{+}W^{-}\) pair, which decay semileptonically to a \(q\overline{q}'\ell\nu\) final state. The dark matter is expected to pass invisibly through the detector, resulting in an imbalance of momentum in the plane transverse to the beam line. Candidate events are selected on the basis of missing transverse momentum recoiling against a boosted system of jets and one lepton. Contributions to the selected events from Standard Model processes are evaluated using Monte Carlo simulation, and data-driven methods are used to constrain the contributions from dominant \(W(\ell\nu)\)+jets and \(\ttbar(WbWb)\) processes. The search design is optimized to test the Dark Higgs model, which predicts a signature of dark matter production in association with the emission of a hypothesized ``Dark Higgs" boson in the dark sector, which decays to a pair of Standard Model particles via a small mixing with the Standard Model Higgs boson. The final search is performed by looking for a deviation from the Standard Model background processes in the distribution of reconstructed Dark Higgs masses for the selected events. New parameter space is probed for Dark Higgs model, complementing existing searches which targeted alternative final states. 



